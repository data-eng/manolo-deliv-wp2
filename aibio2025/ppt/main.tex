%----------------------------------------------------------------------------------------
%	PACKAGES AND THEMES
%----------------------------------------------------------------------------------------
\documentclass[aspectratio=169,xcolor=dvipsnames, t]{beamer}
\usepackage{fontspec}
\usetheme{SimplePlusAIC}
\usepackage{hyperref}
\usepackage{graphicx}
\usepackage{booktabs}
\usepackage{svg}
\usepackage{tikz}
\usepackage{makecell}
\usepackage{wrapfig}


% Used to v-align the logos in the \date line
\newcommand{\vcenteredlogo}[1]{\begingroup
\setbox0=\hbox{\includegraphics[height=20pt]{#1}}%
\parbox{\wd0}{\box0}\endgroup}

\newcommand{\textbfburgundy}[1]{\textbf{\textcolor{DarkBurgundy}{#1}}}


%----------------------------------------------------------------------------------------
%	TITLE PAGE CONFIGURATION
%----------------------------------------------------------------------------------------

\title[short title]{xSTAE:}
\subtitle{Explaining Classifier Decisions through EEG Signal Style Transfer Autoencoding}

\author{\vskip -15pt Natalia Koliou (presenter),
  \\ Panagiotis Zazos, Christoforos Romesis, Cristian Bosch,
  \\ Stasinos Konstantopoulos, and Panagiotis Trakadas}
\institute[\textit{xSTAE: Explaining Classifier Decisions through EEG Signal Style Transfer Autoencoding}]{}

\date{%
  \raisebox{-0.2\height}{\includegraphics[width=2cm]{AICStyleData/logos/fdi.png}}%
  \hspace{2em}%
  \vcenteredlogo{AICStyleData/logos/ceadar.png}%
  \hspace{2em}%
  \vcenteredlogo{AICStyleData/logos/manolo-tagline.png}%
  \hspace{2em}%
  \vcenteredlogo{AICStyleData/logos/euflag-white.png}%
  \hfill 25 Oct 2025 \hspace*{0.15\linewidth}%
}

%----------------------------------------------------------------------------------------
%	PRESENTATION SLIDES
%----------------------------------------------------------------------------------------

\begin{document}

\maketitlepage

%------------------------------------------------
% Overview
\begin{frame}[t]{Overview}
    \tableofcontents
\end{frame}

%------------------------------------------------
% Introduction & Related Work
\makesection{Introduction \& Related Work}

\begin{frame}{Introduction \& Related Work}
    \begin{itemize}
        \item Understanding EEG-based sleep stage classification is crucial for sleep research and healthcare.
        \item Traditional models struggle with \textbfburgundy{misclassifications}, especially for minority sleep stages.
        \item Existing work interprets time-series classifiers, either by integrating explainers into the model (e.g., XTF-CNN, DeepVix) or using post-hoc techniques to highlight which input patterns drive predictions (e.g., timeXplain, LIME).
        \item \textbfburgundy{Counterfactual explanations} have been successful in visual domains but are underexplored for time-series EEG.
        \item We address this gap by generating \textbfburgundy{counterfactual EEG examples} that show what a misclassified instance should have looked like to be correctly classified.

    \end{itemize}
\end{frame}

%------------------------------------------------
% Problem Statement
\makesection{Problem Statement}

\begin{frame}{Problem Statement}
    \begin{itemize}
        \item Counterfactual explanations provide \textbfburgundy{instance-based insights} for time-series classifiers.
        \item The goal is to explain why a classifier predicts a certain label:
        \begin{itemize}
            \item Dataset: $D = \{x^{(i)}\}_{i=1}^N$, each $x^{(i)} \in \mathbb{R}^d$.
            \item Classifier: $C: \mathcal{X} \rightarrow \mathcal{Y}$, labels $\mathcal{Y} = \{1, \dots, n\}$.
            \item For input $x$, predicted label $y = C(x)$.
        \end{itemize}
        \item For misclassified instances $x_m$ with $C(x_m) \neq y^*(x_m)$, find \textbfburgundy{minimal modification} $x'_m$ such that $C(x'_m) = y^*(x_m)$ and $x'_m$ is close to $x_m$.
        \item $\Delta x_m = x'_m - x_m$ reveals dominant patterns influencing the classifier's decision.
    \end{itemize}
\end{frame}

%------------------------------------------------
% Proposed Methodology
\makesection{Proposed Methodology}

\begin{frame}{Proposed Methodology}
    \begin{itemize}
        \item \textit{\textbfburgundy{xSTAE}}: A generative framework using class-conditional \textbfburgundy{autoencoders}.
        \item Each autoencoder $E_\text{tgt}: \mathcal{X} \rightarrow \mathcal{X}_\text{tgt}$ is trained to reconstruct any input while restyling it toward a target class $y_\text{tgt} \in \mathcal{Y}$.
        \item For input $x$ with $y_\text{tgt} \neq C(x)$, the autoencoder generates a counterfactual:
        \[
            x' = E_\text{tgt}(x) \quad \text{s.t.} \quad C(x') = y_\text{tgt}
        \]
        \item Comparing $x$ and $x'$ reveals the patterns in $x$ responsible for the classifier’s original decision.
        \item Training uses dual loss function:
        \begin{enumerate}
            \item \textbfburgundy{Identity:} Ensures $x'$ remains similar to $x$, using a distance function $d(x, x')$.
            \item \textbfburgundy{Classification:} Ensures $x'$ is classified as the target, by comparing $C(x')$ with $y_\text{tgt}$.
        \end{enumerate}
    \end{itemize}
\end{frame}

\begin{frame}{Proposed Methodology}
    \vspace{-1.5em}
    \begin{figure}
        \centering
        \includegraphics[height=0.68\paperheight]{figures/xstae_arch.png}
    \end{figure}
\end{frame}


%------------------------------------------------
% Experiments
\makesection{Experiments}

%------------------------------------------------
% Setup
\begin{frame}{Setup}

    We use the \textbfburgundy{BOAS EEG dataset}, with 128 full-night recordings from 2 channels (256 Hz), to explain classifier errors across 4 sleep stages (N1, N2, N3, REM).
    \vspace{1.5em}

    \begin{itemize}
        \item \textbfburgundy{EEG Data:} 
        \begin{itemize}
            \item 30s epochs ($n_s = 7680$ samples, $n_c = 2$).
            \item FFT $\rightarrow$ 0.4-30 Hz filtering $\rightarrow$ segment into $n_s'=300$ spectral slices.
            \item Extract 3 features per slice (frequency, phase, amplitude).
        \end{itemize}
        \item \textbfburgundy{Classifier:} Two-stage convolutional network over sequences of $k=5$ epochs:
        \[
            X_t^f = [e_{t-k+1}, \dots, e_t] \rightarrow \hat{y}_t
        \]
        \item \textbfburgundy{Autoencoders:} Hybrid networks with convolutional and attention layers:
        \[
        X_f^t = e_t \oplus \text{pos} \longrightarrow \hat{X}_f^{\,t} \text{ (reconstructed input)}
        \]
    \end{itemize}

\end{frame}

%------------------------------------------------
% Quantitative Results
\begin{frame}{Quantitative Results}
    \begin{itemize}
        \item We pass the test dataset through each class-specific autoencoder, then classify the restyled outputs using the original classifier.
        \item High \textbfburgundy{classification accuracy} on restyled signals indicates the autoencoders have learned the patterns that the classifier uses for each sleep stage.
    \end{itemize}

    \vspace{2em}
    \centering
    \begin{tabular}{lcccc}
        \toprule
        Metric & N1 & N2 & N3 & REM \\
        \midrule
        Accuracy & 0.9997 & 0.9998 & 0.9260 & 0.9986 \\
        \bottomrule
    \end{tabular}
\end{frame}


%------------------------------------------------
% Qualitative Discussion
\begin{frame}{Qualitative Discussion}
    \begin{itemize}
        \item \textbfburgundy{Example case:} Original EEG segment was labeled as N2 but misclassified as N1.
        \item xSTAE restyles the segment toward the correct class (N2), producing a counterfactual.
        \item Comparison reveals why the classifier erred:
        \begin{itemize}
            \item Original signal lacked sufficiently prominent spikes.
            \item Restyled signal emphasizes features around timesteps 55-60, 70-75, 90-95.
            \item Classifier "expects" these more pronounced spikes to identify N2.
        \end{itemize}
        \item Misclassification occurs not from wrong feature detection, but from \textbfburgundy{insufficient weighting} of existing patterns.
    \end{itemize}
\end{frame}

\begin{frame}{Qualitative Discussion}
    \vspace{-1.5em}
    \begin{figure}
        \centering
        \includegraphics[width=0.56\textwidth]{figures/org0_tgt1_s3_f1a.png}
    \end{figure}
\end{frame}

%------------------------------------------------
% Conclusion & Future Work
\makesection{Conclusion \& Future Work}

\begin{frame}{Conclusion}
\begin{itemize}
    \item \textbfburgundy{xSTAE} explains classifier errors by restyling misclassified EEG instances into correctly classified ones.
    \item Autoencoders balance \textbfburgundy{identity loss} (keep instance similar) and \textbfburgundy{classification loss} (push to correct label), revealing features the classifier missed.
    \item Contributions:
    \begin{itemize}
        \item Ground counterfactual explanations in time-series EEG classification.
        \item Identify spectral representations suitable for EEG signals.
        \item Validate on open BOAS dataset and release full experimental setup.
    \end{itemize}
\end{itemize}
\end{frame}

\begin{frame}{Future Work}
\begin{itemize}
    \item Explore \textbfburgundy{alternative identity losses} to highlight meaningful changes (e.g., bigger local changes, selective brainwave bands).
    \item Conduct \textbfburgundy{expert trials} to refine interpretability of restyled EEGs.
    \item Investigate linking insights from misclassifications to \textbfburgundy{actionable guidance} at the data or confidence level, keeping xSTAE model-agnostic.
\end{itemize}
\end{frame}

%------------------------------------------------
% Citations
% \begin{frame}{References}
%     % Beamer does not support BibTeX so references must be inserted manually as below
%     \footnotesize{
%         \begin{thebibliography}{99}
%             \bibitem[Smith, 2012]{p1} John Smith (2012)
%             \newblock Title of the publication
%             \newblock \emph{Journal Name} 12(3), 45 -- 678.

%             \bibitem[Doe, 2012]{p1} Joe Doe (2012)
%             \newblock Title of the publication
%             \newblock \emph{Journal Name} 12(3), 45 -- 678.
%             \bibitem[Doe, 2013]{p} Jane Doe (2012)
%             \newblock Title of the publication
%             \newblock \emph{Journal Name} 12(3), 45 -- 678.
%         \end{thebibliography}
%     }
% \end{frame}

%----------------------------------------------------------------------------------------
% Final PAGE
% Set the text that is showed on the final slide
\finalpagetext{Thank you for your attention!}
%----------------------------------------------------------------------------------------
\makefinalpage
%----------------------------------------------------------------------------------------
\end{document}
